\section{Briefing}
Come ogni lavoro che richiede la cooperazione di pi� persone � stata necessaria una fase di briefing preliminare, al fine di concordare le linee guida del lavoro.\\
Poich� il gruppo � composto da due sole persone, e per mancanza della disponibilit� di una persona esterna, si � deciso di non avvalersi della figura dell'osservatore. Tale decisione stata d'obbligo, poich� in base al grafico riportato in fig. \ref{fig:grafico_curva_errori}, con un solo esaminatore si andrebbe a individuare solamente un 25\% dei problemi totali; un valore troppo basso per poter essere significativo. Con due esaminatori si � potuto innalzare tale valore a circa il 50\%, il che rende assai pi� valida e affidabile la valutazione euristica.\\
La decisione ha comportato, inoltre, un'ottimizzazione dei tempi, in quanto entrambi i membri hanno potuto lavorare in autonomia, dedicando alla valutazione euristica parte del tempo libero dai vari impegni personali. Per cercare di essere il pi� uniformi possibile e ridurre cos� il tempo perso per la fusione delle due liste, � stato deciso anche di stilare una sorta di modulo da compilare per ogni problema riscontrato.\\
Di seguito si trova l'elenco delle colonne che componevano tale modulo:
\begin{itemize}
\item Numero progressivo: al fine di effettuare il conteggio delle problematiche riscontrate pi� rapido, ed ottenere cos� delle statistiche pi� rapide e precise;
\item Titolo significativo: per velocizzare il confronto e la ricerca durante la fase di unificazione;
\item Descrizione: qualche riga che descrivesse il problema e in alcuni casi commenti personali dell'esaminatore;
\item Principi violati: un elenco dei principi che sono violati, i quali possono essere anche pi� di uno, poich� uno stesso problema pi� interessare pi� aree;
\item Criticit�: un numero che indica la gravit� di tale problema; nel caso di problemi che interessano pi� categorie si � convenuto di esprimere un solo livello di criticit� che ovviamente sar� pi� alto poich� trasversale a diverse aree.
\end{itemize}
\subsection{Principi adottati}
Come anticipato nell'introduzione, i principi che il nostro gruppo ha deciso di adottare per la valutazione euristica, sono quelli enunciati da Nielsen. Le ragioni di tale scelta sono:
\begin{itemize}
\item Elasticit�: i principi di Nielsen sono stati pensati ed elaborati al fine di poter valutare la maggior parte delle applicazioni odierne; tutti gli aspetti strutturali dell'applicazione hanno lo stesso peso, pertanto si opera un'analisi completa sotto ogni punto di vista
\item Popolarit�: essendo tali principi proposti ed usati da quello che si pu� definire il fondatore della disciplina che studia ``l'usabilit�'', essi sono largamente utilizzati, pertanto le molteplici valutazioni fatte su diverse applicazioni che lavorano con il medesimo obiettivo, si prestano a un pi� facile confronto
\item Disponibilit�: non usare i principi di Nielsen avrebbe anche comportato il fatto che avremmo dovuto  pensarne di nuovi, argomentarli e studiarli a lungo al fine di renderli efficienti ed efficaci
\end{itemize}
Di seguito possiamo trovare i 10 principi usati:
\begin{enumerate}
\item Far vedere lo stato del sistema (feedback)
\item Adeguare il sistema al mondo reale (parlare il linguaggio dell'utente)
\item Controllo dell'utente e libert� (uscite indicate chiaramente)
\item Assicurare consistenza (nell'applicazione, sistema, ambiente)
\item Riconoscimento piuttosto che uso della memoria dell'utente
\item Assicurare essibilit� ed efficienza d'uso (acceleratori)
\item Visualizzare tutte e sole le informazioni necessarie
\item Prevenire gli errori
\item Permettere all'utente di correggere gli errori e non solo di rivelarli
\item Help e documentazione
\end{enumerate}
Com'� stato detto in precedenza, il nostro modulo per la catalogazione dell'errore, prevedeva anche l'inserimento di un livello di criticit�, e anche in questo caso abbiamo deciso di usare la scala suggerita dallo stesso Nielsen, per un discorso di uniformit�:\\
{\bf 0} = Non sono d'accordo che questo sia un problema di usabilit�\\
{\bf 1} = \'E solo un problema ``cosmetico'' (accessorio): non deve essere risolto, a meno che nel progetto non sia disponibile del tempo extra\\
{\bf 2} = Problema secondario: alla sua risoluzione bisognerebbe dare bassa priorit�\\
{\bf 3} = Problema rilevante: � importante risolverlo, bisognerebbe dare alta priorit� alla sua risoluzione\\
{\bf 4} = Catastrofe di usabilit�: � imperativo risolverlo prima che il prodotto
possa essere rilasciato\\


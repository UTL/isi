\section{Debriefing}
Questa fase � necessaria per unire i problemi riscontrati dai due componenti del gruppo, verificare la presenza di eventuali doppioni e infine uniformare i livelli di criticit� qualora entrambi gli esaminatori avessero individuato lo stesso problema di usabilit�. Durante questa fase abbiamo anche di stilato una serie di grafici statistici, utili per effettuare confronti, ma per tali riflessioni rimandiamo alla sezione \ref{analisi_dati}.\\
Il debriefing non ha necessitato di alcun tipo di organizzazione poich� non � stata altro ch una riunione fra i membri del gruppo durante la quale ognuno, a turno, esponeva i problemi individuati valutando se tale poblema era stato individuato anche dall'altro esaminatore, e nel qual caso aveva luogo una breve discussione e si uniformavano i giudizi.\\
Essendo il gruppo composto da solamente due persone ha richiesto un lasso di tempo operativo relativamente breve.

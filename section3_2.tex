\section{Documento presentato agli utenti}
Come detto nei precedenti paragrafi il documento presentato agli utenti � stato scritto in modo da mettere l'utente quanto pi� a suo agio possibile. Per perseguire questo fine abbiamo deciso di scrivere il documento in maniera informale, tenendo conto anche del fatto che gli utenti sono persone conosciute dai valutatori, sarebbe stato perci� una forzatura rivolgersi ad essi in maniera formale.\\
\newline
Di seguito � riportato il documento sottoposto agli utenti:\\
\newline
\newline
Esperimento - Valutazione sito www.dell.it\\
\newline
Innanzitutto ti ringraziamo per la partecipazione.\\
\newline
Lo scopo di questo test � quello di valutare l'usabilit� del sito www.dell.it  e non giudicare le tue abilit� nel portare a termine i vari compiti.\\
Affronta i vari quesiti con calma, senza preoccuparti nel caso in cui tu non riesca a portare a termine qualche compito.\\
\newline
Istruzioni per lo svolgimento:
\begin{itemize}
\item cerca di svolgere i compiti riportati di seguito uno alla volta, nell'ordine in cui sono elencati;
\item quando giungi al termine di ogni compito indica il grado di difficolt� che hai incontrato durante lo svolgimento;
\item quando hai terminato tutti i compiti avverti il valutatore;
\item prima dell'inizio dei vari compiti puoi chiedere chiarimenti, ma non suggerimenti su come portarlo a termine.
\end{itemize}
Ti ricordiamo ancora una volta che lo scopo di questo test � valutare il sito Dell, non le tue capacit�. Ora puoi iniziare a leggere i quesiti, se hai dei dubbi sentiti libero di chiedere chiarimenti al valutatore.\\
\begin{enumerate}
\item {\it Accedi al sito www.dell.it};\\
Indica come hai trovato questo compito:\\
-banale \hspace{1cm} -facile \hspace{1cm} -medio \hspace{1cm} -impegnativo \hspace{1cm} -difficile 
\item {\it Visualizza i prodotti Blu-Ray};\\
Indica come hai trovato questo compito:\\
-banale \hspace{1cm} -facile \hspace{1cm} -medio \hspace{1cm} -impegnativo \hspace{1cm} -difficile 
\item 1. Confronta i due notebook ``Inspiron 15R'' e ``Inspiron M501R'';\\
Indica come hai trovato questo compito:\\
-banale \hspace{1cm} -facile \hspace{1cm} -medio \hspace{1cm} -impegnativo \hspace{1cm} -difficile
\item {\it Registra un nuovo account};\\
Indica come hai trovato questo compito:\\
-banale \hspace{1cm} -facile \hspace{1cm} -medio \hspace{1cm} -impegnativo \hspace{1cm} -difficile 
\item {\it Trova i drivers disponibili per il laptop ``Studio XPS Laptop 1645''};\\
Indica come hai trovato questo compito:\\
-banale \hspace{1cm} -facile \hspace{1cm} -medio \hspace{1cm} -impegnativo \hspace{1cm} -difficile
\item {\it Aggiunta al carrello del computer}, ``Inspiron M501R'' con le seguenti caratteristiche:
\begin{itemize}
\item Processore AMD Phenom II Triple-Core N850 
\item Microsoft Office 2007 Home and Student
\end{itemize}
Indica come hai trovato questo compito:\\
-banale \hspace{1cm} -facile \hspace{1cm} -medio \hspace{1cm} -impegnativo \hspace{1cm} -difficile
\item {\it Effetuare il logout senza chiudere il browser};\\
Indica come hai trovato questo compito:\\
-banale \hspace{1cm} -facile \hspace{1cm} -medio \hspace{1cm} -impegnativo \hspace{1cm} -difficile
\item {\it Aggiungi al carrello un computer desktop ``Dell Studio 15'' con le seguenti caratteristiche}:
\begin{itemize}
\item Intel core i3 Windows 7 Home Premium;
\item Colore Blu ;
\item Microsoft Office 2010 Professional;
\item Supporti di ripristino del sistema operativo: DVD di risorse di Windows 7 Home Premium;
\item Quattro anni di supporto hardware entro il giorno lavorativo successivo, con protezione contro danni accidentali;
\item Backup Online 50Gb;
\item Valigetta;
\item Unit� di storage esterna di almeno 250Gb;
\item Alimentatore di riserva;
\end{itemize}
Indica come hai trovato questo compito:\\
-banale \hspace{1cm} -facile \hspace{1cm} -medio \hspace{1cm} -impegnativo \hspace{1cm} -difficile
\item {\it Salva il portatile appena aggiunto al carrello nell'account che hai creato prima};\\
Indica come hai trovato questo compito:\\
-banale \hspace{1cm} -facile \hspace{1cm} -medio \hspace{1cm} -impegnativo \hspace{1cm} -difficile
\item {\it Elimina il prodotto ``Inspiron M501R'' dal carrello}.\\
Indica come hai trovato questo compito:\\
-banale \hspace{1cm} -facile \hspace{1cm} -medio \hspace{1cm} -impegnativo \hspace{1cm} -difficile
\end{enumerate}

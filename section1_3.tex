\section{Definizione del profilo utente}
Scopo del presente paragrafo � delineare gli attributi di un probabile utente
utilizzatore del sistema in esame. Questa � una fase cruciale di ogni analisi di usabilit�, infatti le assunzioni qui prese condizioneranno i successivi metodi di valutazione.\\
Avvalendoci del nostro buon senso, abbiamo stilato un profilo psicologico dell'utente medio che accede al sito web; � stato valutato e consultato il forum di supporto, tuttavia essendo rivolto esclusivamente a coloro che hanno dei problemi, non ci d� informazioni utili: coloro che discutono nel forum sono persone generalmente abituate a una forte autonomia, mentre utenti novizi, a fronte di un problema � pi� facile che cerchino l'aiuto di una persona esperta, come un amico, il quale poi successivamente chieder� aiuto sul forum.\\
In base a tale ragionamento abbiamo provato a stilare un elenco di caratteristiche che delineano la classe di utenti/clienti del portale:\\

\smallskip

{\bf \noindent Caratteristiche Psicologiche}
\begin{itemize}
\item Stile cognitivo: il sistema permette a chi ha delle conoscenze pregresse di sfruttarle, mentre coloro che sono meno esperti vengono supportati dal sistema: mostra le procedure sempre uguali e, per persone che sono pi� intuitive, il sito web fornisce alternative al percorso tradizionale
\item Attitudine (dell'utente rispetto al sistema): positiva, gi� il fatto che
l'utente tenti di comprare via internet indica che � ben disposto
all'utilizzo del sito. Ovviamente avendo il sito web anche la funzione di vetrina, risulta essere adatto anche a coloro che vogliono solamente farsi un'idea delle soluzioni informatiche presenti sul mercato 
\item Motivazione (dell'utente rispetto all'uso): alta, in quanto non � possibile acquistare in un negozio tradizionali i prodotti Dell
\end{itemize}

{\bf \noindent Conoscenze}
\begin{itemize}
\item Livelli di alfabetizzazione (lettura): l'utente deve saper leggere e comprendere le istruzioni che gli vengono visualizzate sullo schermo
\item Titolo di studio: minimo, licenza media
\item Abilit� dattilografica: bassa, in quanto l'interazione con il portale necessita in larga parte dell'utilizo del mouse, mentre la tastiera � utilizzata solo per funzionalit� specifiche e molto limite
\item Linguaggio: in massima parte il sito web risulta essere tradotto in molteplici lingue di tutto il mondo, anche se verr� specificato nei capitoli successivi, vi sono ancora determinate parti rimaste in inglese
\item Alfabetizzazione informatica: moderata, l'utente � capace di navi-
gare su internet, quindi, conosce i rudimenti dell'informatica. Inoltre per certe funzionalit� deve saper riconoscere alcuni componenti di un computer, ma non � un requisito fondamentale
\end{itemize}
{\bf \noindent Esperienza}
\begin{itemize}
\item Uso dei sistemi informatici: media, in quanto l'utente deve saper navigare nel web e deve avere un'esperienza positiva con il pagamento tramite carte di credito o prepagata. Un utente novizio, con scarsa esperienza nell'utilizzo di tali strumenti potrebbe avere dei dubbi sulla sicurezza della metodologia di pagamento e sull'affidabilit� dei corrieri
\item Dell'applicazione: nessuna, in quanto il sito internet si propone con finestre intuitive che consentono anche ad utenti meno esperti di poter usufriuire del portale con successo 
\end{itemize}
{\bf \noindent Caratteristiche fisiche dell'utente}
\begin{itemize}
\item Vista: requisito fondamentale � che l'utente non sia cieco, in quanto il sistema non prevede una navigazione per non i vedenti (non rispetta gli standard del W3C. Per quanto riguarda il resto dei difetti, non vi sono problemi, poich� un eventuale daltonismo non comporta alcun tipo di limitazione
\item Manualit�: non vi � nessun vincolo, basta poter usare la tastiera  il mouse, o periferiche sostitutive alle due citate
\item Maggioreit�: per poter effettuare un ordine valido � necessario essere maggiorenni, in quanto il pagamento elettronico o il bonifico bancario richiedono tale caratteristica; da sottolineare che la registrazione al sito della Dell, non effettua alcun tipo di controllo sull'et�, pertanto si potrebbe ordinare un prodotto anche senza essere maggiorenni
\end{itemize}
{\bf \noindent Caratteristiche sociali}
\begin{itemize}
\item Tipologia di lavoro: nessun requisito, poich� per acquistare basta essere maggiorenni
\item Frequenza di turn-over: alta, in quanto quasi sempre nuovi gli utenti che acquistano, vista anche la durabilit� del bene acquistato
\item Importanza del compito: bassa in genere, serve solo per soddisfare un bisogno dell'utente; se viene incluso l'accesso alla sezione del supporto allora la frequenza del turno over pu� essere considerata moderata.
\item Frequenza d'uso: estremamente bassa, poich� salvo ulteriori acquisti o la necessit� di richiedere assistenza, l'utente potrebbe anche non aver pi� bisogno di usare il portale
\item Addestramento di base: nessuno, il sito dovrebbe essere autoesplicativo
\end{itemize}
Dall'analisi appena effettuata possiamo dire che la popolazione d'utenti utlizzatori del sito � altamente variagata, se poi si pensa che il sito web si compone anche delle sezioni per le aziende e per la pubblica ammistrazione, la variet� di tipologie d'utenti aumenta notevolmente, considerando aspetti pi� professionali e meno legati al puro svago.\\
Nei capitoli che seguono abbiamo utilizzato il profilo appena tracciato per effettuare la valutazione.
